\documentclass{article}
\usepackage{amsmath,amssymb,amsthm} % AMS styles for extra equation formatting
\usepackage{graphicx} % for including graphics files
\usepackage{subfig} % for subfigures
\usepackage[numbers,sort]{natbib} % for better references control
\usepackage{hyperref} % for hyperlinks within the paper and references
\usepackage{fontspec}  % Allows for system fonts
\usepackage[top=2cm, bottom=2cm, left=2cm, right=2cm]{geometry}  % Set margins on all sides
\usepackage{setspace} % for line spacing
\usepackage{appendix} % for the appendices
\usepackage{listings} % for code
\usepackage{xcolor} % for color
\usepackage{url,textcomp}
\usepackage{matlab-prettifier}
%%%%%%%%%%%%%%%%%%%%%%%%%%%%%%%%%%%%%%%%%%%%%%%%%%%%%%%%%%%%%%%%%%%%%%%%%%%%%%

\hypersetup{colorlinks=true, linkcolor=blue,  anchorcolor=blue,
citecolor=blue, filecolor=blue, menucolor=blue, pagecolor=blue,
urlcolor=blue}

%%%%%%%%%%%%%%%%%%%%%%%%%%%%%%%%%%%%%%%%%%%%%%%%%%%%%%%%%%%%%%%%%%%%%%%%%%%%%%

\newcommand{\todo}[1]{\vspace{5 mm}\par \noindent
\marginpar{\textsc{Todo}}
\framebox{\begin{minipage}[c]{0.90 \textwidth}
\tt \flushleft #1 \end{minipage}}\vspace{5 mm}\par}
\newcommand{\setParDis}{\setlength {\parskip} {0.2cm} } % for 0.3cm spacing
\newcommand{\setParDef}{\setlength {\parskip} {0pt} } % for 0 spacing

%%%%%%%%%%%%%%%%%%%%%%%%%%%%%%%%%%%%%%%%%%%%%%%%%%%%%%%%%%%%%%%%%%%%%%%%%%%%%%

\graphicspath{{graphics/}}

\newtheorem{theorem}{Theorem}[section]
\newtheorem{proposition}[theorem]{Proposition}
\newtheorem{lemma}[theorem]{Lemma}
\newtheorem{corollary}[theorem]{Corollary}
\newtheorem{definition}[theorem]{Definition}

%\renewcommand{\qedsymbol}{$\blacksquare$} % for filled square at end of proof
%\numberwithin{equation}{section} % for the 1.1, 1.2 equation number style
%\setlength{\parindent}{0em} % don't indent paragraphs
%\setlength{\parskip}{1em} % add spacing between paragraphs
%\linespread{1.6} % double-spacing

\setmainfont{Arial}
% \doublespacing
\onehalfspacing
\setcounter{secnumdepth}{3}

%%%%%%%%%%%%%%%%%%%%%%%%%%%%%%%%%%%%%%%%%%%%%%%%%%%%%%%%%%%%%%%%%%%%%%%%%%%%%%

\begin{document}

% \title{This is the title}
% \author{A.N. Author and A. Friend}
% \date{\today}
% \maketitle

\begin{titlepage}
  \centering  % Center everything on the title page
  \vspace*{\fill}  % Add flexible vertical space at the top to push the title down

  {\Huge\bfseries Visual Search of an Image Collection}  % Set the title in large, bold font
  \vskip 0.3em  % Add some space between title and author

  {\Large\itshape EEE3032 - Computer Vision and Pattern Recognition \\
  Coursework Assignment}  % Set the author name in a slightly smaller font
  \vskip 1em  % Add space between author and date
  
  {\normalsize\slshape Xiaoguang Liang}  % Set the author name in a slightly smaller font
  % \vskip 0.1em  % Add space between author and date

  {\normalsize\slshape 6844178}  % Set the author name in a slightly smaller font
  % \vskip 0.1em  % Add space between author and date

  {\normalsize\slshape xl01339@surrey.ac.uk}  % Set the author name in a slightly smaller font
  \vskip 1em  % Add space between author and date
  
  {\normalsize\slshape \today}  % Set the date in a smaller font
  
  \vspace*{\fill}  % Add flexible vertical space at the bottom to center the content
\end{titlepage}

% Suppress any floats (figures, tables) from appearing on the next page
\suppressfloats

\tableofcontents

\begin{abstract}
This project aims to explore different algorithms for visual searching of an image collection. For each algorithm, a feature database is constructed by computing descriptors, and a test image is used as a query to return a list of the top N images that best match the query by calculating similarity. The project tests and optimizes the performance of each algorithm by adjusting parameters and evaluates the search results using PR curves. Additionally, a comparative evaluation of the performance of different algorithms is conducted.
\end{abstract}

%%%%%%%%%%%%%%%%%%%%%%%%%%%%%%%%%%%%%%%%%%%%%%%%%%%%%%%%%%%%%%%%%%%%%%%%%%%%%%

\section{Introduction}
\setParDis

Compared to traditional text-based image search, visual appearance-based search is more effective at describing the visual characteristics of an image, while text is only suitable for describing the objects present in the image. This report explores various visual search algorithms using the MSRC-v2 image database, including Global Colour Histogram, Spatial Grid, PCA, and SIFT. Additionally, this project investigates the classification task on the current image dataset using BoVW (Bag of Visual Words) and SVM based on the computed descriptors. Figure 1 shows the project architecture.

Figure 1 shows the project architecture. The implementation details are structured as follows: \textit{Section 2} covers the implementation of visual search techniques. \textit{Section 3} delves into experiments using various descriptors and distance measures. In \textit{Section 4}, the report outlines the implementation of a basic BoVW system. \textit{Section 5} focuses on classifying the categories within the image collection. Finally, conclusions are drawn in \textit{Section 6}. The entire project was implemented using Matlab.

% \begin{figure}[ht]
% \begin{center}
% \includegraphics[width=12cm]{Block diagram of the simplified source-filter model of speech production}
% \end{center}
% \caption{\label{fig:source-filter} Block diagram of the simplified source-filter model of speech production\citep{kondoz2005digital}.}
% \end{figure}


% \begin{figure}[ht]
% \begin{center}
% \includegraphics[width=18cm]{The framework of implementation of model estimation and synthesis}
% \end{center}
% \caption{\label{fig:framework} The framework of implementation of model estimation and synthesis.}
% \end{figure}

%%%%%%%%%%%%%%%%%%%%%%%%%%%%%%%%%%%%%%%%%%%%%%%%%%%%%%%%%%%%%%%%%%%%%%%%%%%%%%

\section{Implementation of Visual search techniques}

\subsection{Algorithms for computing descriptors}



\subsubsection{Global colour histogram}



\subsubsection{Spatial grid}



\subsubsection{PCA}



\subsubsection{SIFT}



\subsection{Evaluation methodology}




\section{Experiments}

\subsection{Experimental design}

This experiment uses the variable-controlling approach to examine the impact of different order values and segment lengths on the LPC frequency response and the synthesis speech quality.


\subsection{Different descriptors}


\subsection{Different distance measures}



\section{BoVW}


\section{Classify with SVM}


\section{Conclusion}


%%%%%%%%%%%%%%%%%%%%%%%%%%%%%%%%%%%%%%%%%%%%%%%%%%%%%%%%%%%%%%%%%%%%%%%%%%%%%%

\newcommand{\doi}[1]{DOI: \href{http://dx.doi.org/#1}{\nolinkurl{#1}}}
\bibliographystyle{unsrt}
\bibliography{refs}

%%%%%%%%%%%%%%%%%%%%%%%%%%%%%%%%%%%%%%%%%%%%%%%%%%%%%%%%%%%%%%%%%%%%%%%%%%%%%%

\end{document}
